%!TEX root = ../template.tex
%%%%%%%%%%%%%%%%%%%%%%%%%%%%%%%%%%%%%%%%%%%%%%%%%%%%%%%%%%%%%%%%%%%%
%% chapter4.tex
%% NOVA thesis document file
%%
%% Chapter with lots of dummy text
%%%%%%%%%%%%%%%%%%%%%%%%%%%%%%%%%%%%%%%%%%%%%%%%%%%%%%%%%%%%%%%%%%%%


%Implementação

\typeout{NT FILE chapter4.tex}%

\chapter{Conclusion}
\label{cha:conclusion}


With the ongoing development of wearable technology, particularly smart insoles and socks, there is also a significant advancement in basketball motion recognition and 
injury prevention. Even tho wearables have several restraints, this technology is the most appropriate for basketball motion recognition and injury prevention and its biggest step 
back is the prohibition in official matches.

By capturing real-time data on pressure distribution, foot orientation, and movement dynamics, these wearables allow for a 
detailed analysis of key basketball actions like jump shots, layups, pivots, and lateral movements, providing valuable insights into each player's unique biomechanics for the players 
and coaches. 
